%\ifnum\pdfoutput>0 % pdflatex compilation
\documentclass[a4paper,10pt]{scrartcl}
\usepackage[pdftex]{graphicx}
\DeclareGraphicsExtensions{.pdf,.png,.jpg}
\RequirePackage[hyperindex]{hyperref}
%\else % htlatex compilation
%\documentclass{article}
%\usepackage{graphicx}
%\DeclareGraphicsExtensions{.png, .gif, .jpg}
%\newcommand{\href}[2]{\Link[#1]{}{} #2 \EndLink}
%\newcommand{\hypertarget}[2]{\Link[]{}{#1} #2 \EndLink}
%\newcommand{\hyperlink}[2]{\Link[]{#1}{} #2 \EndLink}
%\newcommand{\url}[1]{\Link[#1]{}{} #1 \EndLink}
%\fi



\begin{document}
\title{\texttt{cplint} Manual}

\subtitle{SWI-Prolog Version}

\author{Fabrizio Riguzzi\\
fabrizio.riguzzi@unife.it}

\maketitle


\section{Introduction}


\texttt{cplint} is a suite of programs for reasoning with LPADs/CP-logic programs  \cite{VenVer03-TR}, \cite{VenVer04-ICLP04-IC},
\cite{VenDenBru-JELIA06},
\cite{DBLP:journals/tplp/VennekensDB09}. It contains modules for both inference and learning.

\texttt{cplint} is available in two versions, one for Yap Prolog and one for SWI-Prolog. They differ slightly in the features offered.
This manual is about the SWI-Prolog version. You can find the manual for the Yap version at \url{http://ds.ing.unife.it/~friguzzi/software/cplint/manual.html}.

\section{Installation}
\texttt{cplint} is distributed as a \href{http://www.swi-prolog.org/pack/list?p=cplint}{pack} of \href{http://www.swi-prolog.org/}{SWI-Prolog}. To install it, use
\begin{verbatim}
?- pack_install(cplint).
\end{verbatim}
Moreover, in order to make sure you have a foreign library that matches your architecture, run
\begin{verbatim}
?- pack_rebuild(cplint). 
\end{verbatim}


\section{Syntax}
\label{syn}
\texttt{cplint} permits the definition of discrete probability distributions and continuous probaility
densities.
\subsection{Discrete Probability Distributions}
\label{discrete}
LPAD and CP-logic programs consist of a set of annotated disjunctive clauses.
Disjunction in the head is represented with a semicolon and atoms in the head are separated from probabilities by a colon. For the rest, the usual syntax of Prolog is used.
A general CP-logic clause has the form
\begin{verbatim}
h1:p1 ; ... ; hn:pn :- Body.
\end{verbatim}
where \verb|Body| is a conjunction of goals as in Prolog.
 No parentheses are necessary. The \texttt{pi} are numeric expressions. It is up to the user to ensure that the numeric expressions are legal, i.e. that they sum up to less than one.

If the clause has an empty body, it can be represented like this
\begin{verbatim}
h1:p1 ; ... ; hn:pn.
\end{verbatim}
If the clause has a single head with probability 1, the annotation can be omitted and the clause takes the form of a normal prolog clause, i.e. 
\begin{verbatim}
h1 :- Body.
\end{verbatim}
stands for 
\begin{verbatim}
h1:1 :- Body.
\end{verbatim}
The coin example of  \cite{VenVer04-ICLP04-IC} is represented as (file \href{http://cplint.eu/example/inference/coin.pl}{\texttt{coin.pl}})
\begin{verbatim}
heads(Coin):1/2 ; tails(Coin):1/2 :- 
  toss(Coin),\+biased(Coin).

heads(Coin):0.6 ; tails(Coin):0.4 :- 
  toss(Coin),biased(Coin).

fair(Coin):0.9 ; biased(Coin):0.1.

toss(coin).
\end{verbatim}
The first clause states that if we toss a coin that is not biased it has equal probability of landing heads and tails. The second states that if the coin is biased it has a slightly higher probability of landing heads. The third states that the coin is fair with probability 0.9 and biased with probability 0.1 and the last clause states that we toss a coin with certainty.

Moreover, the bodies of rules may contain built-in predicates, predicates
from the libraries \verb|lists|, \verb|apply| and \verb|clpr/nf_r|
plus the predicate
\begin{verbatim}
average/2
\end{verbatim}
that, given a list of numbers, computes its arithmetic mean.

The body of rules may also contain the predicate \verb|prob/2| that computes the
probability of an atom, thus allowing nested probability computations.
For example (\href{http://cplint.eu/example/inference/meta.pl}{\texttt{meta.pl}})
\begin{verbatim}
a:0.2:-
  prob(b,P),
  P>0.2.
\end{verbatim}
is a valid rule.

Moreover, the probabilistic annotations can be variables, as in 
(\href{http://cplint.eu/example/inference/flexprob.pl}{\texttt{flexprob.pl}}))
\begin{verbatim}
red(Prob):Prob.

draw_red(R, G):-
  Prob is R/(R + G),
  red(Prob).
\end{verbatim}
Variables in probabilistic annotations must be ground when resolution reaches the end of the body, 
otherwise an exception is raised.

Alternative ways of specifying probability distribution include
\begin{verbatim}
A:discrete(Var,D):-Body.
\end{verbatim}
or
\begin{verbatim}
A:finite(Var,D):-Body.
\end{verbatim}
where \verb|A| is an atom containg variable \verb|Var| and \verb|D|
is a list of couples \verb|Value:Prob| assigning probability \verb|Prob|
to \verb|Value|. 
Moreover, you can use
\begin{verbatim}
A:uniform(Var,D):-Body.
\end{verbatim}
where \verb|A| is an atom containg variable \verb|Var| and \verb|D|
is a list of values each taking the same probability (1 over the length
of \verb|D|).
\subsubsection{ProbLog Syntax}
You can also use ProbLog \cite{DBLP:conf/ijcai/RaedtKT07} syntax, so a general clause takes the form
\begin{verbatim}
p1::h1 ; ... ; pn::hn :- Body
\end{verbatim}
where the \texttt{pi} are numeric expressions. 

\subsubsection{PRISM Syntax}
You can also use PRISM \cite{DBLP:conf/ijcai/SatoK97} syntax, so a program is composed of
a set of regular Prolog rules whose body may contain calls to the \verb|msw/2| predicate (multi-ary 
switch). A call \verb|msw(term,value)| means that a random variable associated to \verb|term|
assumes value \verb|value|. The admissible values  for a discrete random variable are 
specified using facts for the \verb|values/2| predicate of the form
\begin{verbatim}
values(T,L).
\end{verbatim}
where \verb|T| is a term (possibly containing variables) and \verb|L| is a list of values.
The distribution over values is specified using directives for \verb|set_sw/2| of the form
\begin{verbatim}
:- set_sw(T,LP).
\end{verbatim}
where \verb|T| is a term (possibly containing variables) and \verb|LP| is a list of
probability values.
Remember that in PRISM each call to \verb|msw/2| refers to a different random
variable, i.e., no memoing is performed, differently from the case of LPAD/CP-Logic/ProbLog.

For example, the coin example above in PRISM syntax becomes
(\href{http://cplint.eu/example/inference/coinmsw.pl}{\texttt{coinmsw.pl}})
\begin{verbatim}
values(throw(_),[heads,tails]).
:- set_sw(throw(fair),[0.5,0.5]).
:- set_sw(throw(biased),[0.6,0.4]).
values(fairness,[fair,biased]).
:- set_sw(fairness,[0.9,0.1]).
res(Coin,R):- toss(Coin),fairness(Coin,Fairness),msw(throw(Fairness),R).
fairness(_Coin,Fairness) :- msw(fairness,Fairness).
toss(coin).
\end{verbatim}
\subsection{Continuous Probability Densities}
\label{cont}

\verb|cplint| handles continuous random variables as well with its
sampling inference module.
To specify a probability density on an argument \verb|Var| of an atom
\verb|A| you can used rules of the form
\begin{verbatim}
A:Density:- Body
\end{verbatim}
where \verb|Density| is a special atom identifying a probability  density on variable \verb|Var| and \verb|Body| (optional) is a regular clause body.
Allowed \verb|Density| atoms are
\begin{itemize}
\item \verb|uniform(Var,L,U)|: \verb|Var| is uniformly distributed in $[L,U]$
\item \verb|gaussian(Var,Mean,Variance)|: \verb|Var| follows a Gaussian distribution with mean \verb|Mean| and variance \verb|Variance|. The distribution can be multivariate if \verb|Mean| 
is a list and \verb|Variance| a list of lists representing the mean vector and the covariance matrix. In this case the values of \verb|Var| are lists of real values with the same length as
that of \verb|Mean|
\item \verb|dirichlet(Var,Par)|: \verb|Var| is a list of real
numbers following a Dirichlet distribution with $\alpha$ parameters specified
by the list \verb|Par|
\item \verb|gamma(Var,Shape,Scale)|  \verb|Var| follows a gamma distribution 
with shape parameter \verb|Shape| and scale parameter \verb|Scale|.
\item \verb|beta(Var,Alpha,Beta)|  \verb|Var| follows a beta distribution 
with parameters \verb|Alpha| and \verb|Beta|.
\item \verb|poisson(Var,Lambda)|  \verb|Var| follows a Poisson distribution 
with parameter \verb|Lambda|.
\item \verb|binomial(Var,N,P)|  \verb|Var| follows a binomial distribution 
with parameters \verb|N| and \verb|P|.
\item \verb|geometric(Var,P)|  \verb|Var| follows a geometric distribution 
with parameter \verb|P|.
\end{itemize}
For example
\begin{verbatim}
g(X): gaussian(X,0, 1).
\end{verbatim}
states that argument \verb|X| of \verb|g(X)| follows a Gaussian 
distribution with mean 0 and variance 1, while
\begin{verbatim}
g(X): gaussian(X,[0,0], [[1,0],[0,1]]).
\end{verbatim}
states that argument \verb|X| of \verb|g(X)| follows a Gaussian 
multivariate distribution with mean vector $[0,0]$ and covariance matrix
$$\left[\begin{array}{rr}
1&0\\
0&1
\end{array}\right]$$.



For example, \href{http://cplint.eu/example/inference/gaussian_mixture.pl}{\texttt{gaussian\_mixture.pl}} defines a mixture of two Gaussians:
\begin{verbatim}
heads:0.6;tails:0.4.
g(X): gaussian(X,0, 1).
h(X): gaussian(X,5, 2).
mix(X) :- heads, g(X).
mix(X) :- tails, h(X).
\end{verbatim}
The argument \verb|X| of
\verb|mix(X)| follows a distribution that is a mixture of two Gaussian,
one with mean 0 and variance 1 with probability 0.6 and one with 
mean 5 and variance 2 with probability 0.4.

The parameters of the distribution atoms can be taken from the probabilistic
atom, the example \href{http://cplint.eu/example/inference/gauss_mean_est.pl}{\texttt{gauss\_mean\_est.pl}}
\begin{verbatim}
val(I,X) :-
  mean(M),
  val(I,M,X).
mean(M): gaussian(M,1.0, 5.0).
val(_,M,X): gaussian(X,M, 2.0).
\end{verbatim}
states that for an index \verb|I| the continuous variable \verb|X| is 
sampled from a Gaussian whose variance is 2 and whose mean is sampled from a Guassian with mean 1 and
variance 5.

Any operation is allowed on continuous random variables. The example below
(\href{http://cplint.eu/example/inference/kalman_filter.pl}{\texttt{kalman\_filter.pl}}) encodes a Kalman filter:
\begin{verbatim}
kf(N,O, T) :-
  init(S),
  kf_part(0, N, S,O,T).
kf_part(I, N, S,[V|RO], T) :-
  I < N,
  NextI is I+1,
  trans(S,I,NextS),
  emit(NextS,I,V),
  kf_part(NextI, N, NextS,RO, T).
kf_part(N, N, S, [],S).
trans(S,I,NextS) :-
  {NextS =:= E + S},
  trans_err(I,E).
emit(NextS,I,V) :-
  {NextS =:= V+X},
  obs_err(I,X).
init(S):gaussian(S,0,1).
trans_err(_,E):gaussian(E,0,2).
obs_err(_,E):gaussian(E,0,1).
\end{verbatim}
Continuous random variables are involved
in arithmetic expressions (in \verb|trans/3| and \verb|emit/3|). It
is often convenient, as in this case, to use CLP(R) constraints (by
including the directive \verb|:- use_module(library(clpr)).|) as 
in this way the expressions can be used in multiple directions and 
the same clauses can be used both to sample and to evaluate the weight the sample on the basis
of evidence,
otherwise different clauses have to be written.
In case random variables are not sufficiently instantiated to 
exploit expressions for inferring the values of other variables, 
inference will return an error.

\subsubsection{Distributional Clauses Syntax}
\label{dc}
You can also use the syntax of Distributional Clauses (DC) \cite{Nitti2016}.
Continuous random variables are represented in this case by term whose distribution can be specified with density atoms as in
\begin{verbatim}
T~Density' := Body.
\end{verbatim}
Here \verb|:=| replaces the implication symbol, \verb|T| is a term and \verb|Density'| is one of the density atoms above without the \verb|Var| argument, because \verb|T|
itself represents a random variables. In the body of clauses you can use the infix operator \verb|~=| to equate a term representing a random variable with a logical variable or
a constant as in \verb|T ~= X|. Internally \verb|cplint| transforms the terms representing random variables into atoms with an extra argument for holding the variable.
DC can be used to represent also discrete distributions using
\begin{verbatim}
T~uniform(L) := Body.
T~finite(D) := Body.
\end{verbatim} 
where \verb|L| is a list of values and \verb|D| is a list of couples \verb|P:V| with \verb|P| a probability and \verb|V| a value.
If \verb|Body| is empty, as in regular Prolog, the implication symbol \verb|:=| can be omitted.

The Indian GPA problem from \url{http://www.robots.ox.ac.uk/~fwood/anglican/examples/viewer/?worksheet=indian-gpa}in distributional clauses syntax  (\url{https://github.com/davidenitti/DC/blob/master/examples/indian-gpa.pl})
takes the form (\href{http://cplint.eu/example/inference/indian_gpadc.pl}{\texttt{indian\_gpadc.pl}}):
\begin{verbatim}
is_density_A:0.95;is_discrete_A:0.05.
% the probability distribution of GPA scores for American students is
% continuous with probability 0.95 and discrete with probability 0.05

agpa(A): beta(A,8,2) :- is_density_A.
% the GPA of American students follows a beta distribution if the
% distribution is continuous

american_gpa(G) : finite(G,[4.0:0.85,0.0:0.15]) :- is_discrete_A.
% the GPA of American students is 4.0 with probability 0.85 and 0.0
% with 
% probability 0.15 if the
% distribution is discrete
american_gpa(A):- agpa(A0), A is A0*4.0.
% the GPA of American students is obtained by rescaling the value of
% agpa
% to the (0.0,4.0) interval
is_density_I : 0.99; is_discrete_I:0.01.
% the probability distribution of GPA scores for Indian students is
% continuous with probability 0.99 and discrete with probability 
% 0.01
igpa(I): beta(I,5,5) :- is_density_I.
% the GPA of Indian students follows a beta distribution if the
% distribution is continuous
indian_gpa(I): finite(I,[0.0:0.1,10.0:0.9]):-  is_discrete_I.
% the GPA of Indian students is 10.0 with probability 0.9 and 0.0
% with
% probability 0.1 if the
% distribution is discrete
indian_gpa(I) :- igpa(I0), I is I0*10.0.
% the GPA of Indian students is obtained by rescaling the value 
% of igpa
% to the (0.0,4.0) interval
nation(N) : finite(N,[a:0.25,i:0.75]).
% the nation is America with probability 0.25 and India with 
% probability 0.75
student_gpa(G):- nation(a),american_gpa(G).
% the GPA of the student is given by american_gpa if the nation is 
% America
student_gpa(G) :- nation(i),indian_gpa(G).
% the GPA of the student is given by indian_gpa if the nation 
%is India
\end{verbatim}
See 

\section{Semantics}
\label{semantics}

The semantics of LPADs for the case of programs without functions symbols can be given
as follows. An LPAD defines a probability distribution over normal logic programs called
\emph{worlds}. A world is obtained from an LPAD by first grounding it and, by 
selecting a single head atom for each ground clause and by including in the world
the clause with the selected head atom and the body.
The probability of a world is the product of the probabilities associated to the 
heads selected.
The probability of a ground atom (the query) is given by the sum of the probabilities
of the worlds where the query is true.

If the LPAD contains function symbols, the definition is more complex, see
\cite{DBLP:journals/ai/Poole97,DBLP:journals/jair/SatoK01,Rig15-PLP-IW}.
\section{Inference}
\label{inf}
\texttt{cplint} answers queries using the module \verb|pita| or \verb|mcintyre|. The first performs the program transformation technique of \cite{RigSwi10-ICLP10-IC}. Differently from that work, techniques alternative to tabling and answer subsumption are used. The latter performs approximate inference by sampling
using a different program transformation technique and is described in \cite{Rig13-FI-IJ}.

For answering queries, you have to prepare a Prolog file where you first load the inference module (for example \verb|pita|), initialize it with a directive (for example \verb|:- pita|) and then enclose the LPAD
clauses in \verb|:-begin_lpad.| and \verb|:-end_lpad.| For example, the coin program above can be stored in \href{http://cplint.lamping.unife.it/example/inference/coin.pl}{\texttt{coin.pl}} for performing inference
with \verb|pita| as follows
\begin{verbatim}
:- use_module(library(pita)).
:- pita.
:- begin_lpad.
heads(Coin):1/2 ; tails(Coin):1/2:- 
toss(Coin),\+biased(Coin).

heads(Coin):0.6 ; tails(Coin):0.4:- 
toss(Coin),biased(Coin).

fair(Coin):0.9 ; biased(Coin):0.1.

toss(coin).
:- end_lpad.
\end{verbatim}
The same program for \verb|mcintyre| is
\begin{verbatim}
:- use_module(library(mcintyre)).
:- mc.
:- begin_lpad.
heads(Coin):1/2 ; tails(Coin):1/2:- 
toss(Coin),\+biased(Coin).

heads(Coin):0.6 ; tails(Coin):0.4:- 
toss(Coin),biased(Coin).

fair(Coin):0.9 ; biased(Coin):0.1.

toss(coin).
:- end_lpad.
\end{verbatim}
You can have also (non-probabilistic) clauses outside \verb|:-begin/end_lpad.| These are considered as database clauses.
In \verb|pita| subgoals in the body of probabilistic clauses can query them by enclosing the query in \verb|db/1|.
For example (\href{http://cplint.lamping.unife.it/example/inference/testdb.pl}{\texttt{testdb.pl}})
\begin{verbatim}
:- use_module(library(pita)).
:- pita.
:- begin_lpad.
sampled_male(X):0.5:-
  db(male(X)).
:- end_lpad.
male(john).
male(david).
\end{verbatim}
You can also use \verb|findall/3| on subgoals defined by database clauses
(\href{http://cplint.lamping.unife.it/example/inference/persons.pl}{\texttt{persons.pl}})
\begin{verbatim}
:- use_module(library(pita)).
:- pita.
:- begin_lpad.
male:M/P; female:F/P:-
  findall(Male,male(Male),LM),
  findall(Female,female(Female),LF),
  length(LM,M),
  length(LF,F),
  P is F+M.
:- end_lpad.
male(john).
male(david).
female(anna).
female(elen).
female(cathy).
\end{verbatim}
Aggregate predicates on probabilistic subgoals are not implemented due to their high
computational cost (if the aggregation is over $n$ atoms, the possible values of the
aggregation are potentially $2^n$).

In \verb|mcintyre| you can query database clauses in the body of probabilistic clauses without any special syntax. You can also 
use \verb|findall/3|.


Then you can simply load \href{http://cplint.lamping.unife.it/example/inference/coin.pl}{\texttt{coin.pl}} as
\begin{verbatim}
?- [coin].
\end{verbatim}
Note that supplying \href{http://cplint.lamping.unife.it/example/inference/coin.pl}{\texttt{coin.pl}} as an argument to the \verb|swipl| command currently returns errors due to bad interaction between \verb|pita| and the top-level.
The program is loaded correctly anyway but it is recommended to load it from the top-level to avoid these errors.

\subsection{Unconditional Queries}
\label{uncondq}
The unconditional probability of an atom can be asked using \verb|pita| with the predicate
\begin{verbatim}
prob(:Query:atom,-Probability:float).
\end{verbatim}
as in
\begin{verbatim}
?- prob(heads(coin),P).
\end{verbatim}
If the query is non-ground, \verb|prob/2| returns in backtracking the succesful instantiations together with their probability.

When using \verb|mcintyre|, the predicate for querying is
\begin{verbatim}
mc_prob(:Query:atom,-Probability:float,+Options:list).
\end{verbatim} as in
\begin{verbatim}
?- mc_prob(heads(coin),P,[]).
\end{verbatim}
With \verb|mcintyre|, you can also take a given number of sample with
\begin{verbatim}
mc_sample(:Query:atom,+Samples:int,-Probability:float,Options:list).
\end{verbatim}
as in (\href{http://cplint.ml.unife.it/example/inference/coinmc.pl}{\texttt{coinmc.pl}})
\begin{verbatim}
?- mc_sample(heads(coin),1000,P,[successes(S),failures(F)]).
\end{verbatim}
that samples \verb|heads(coin)| 1000 times and returns in \verb|S| the number of successes, in \verb|F| the number of failures and in \verb|P| the
estimated probability (\verb|S/1000|).

Differently from exact inference, in approximate inference the query can be a conjunction of atoms.

If you are just interested in the probability, you can use
\begin{verbatim}
mc_sample(:Query:atom,+Samples:int,-Probability:float,Options:list)
\end{verbatim}
as in (\href{http://cplint.ml.unife.it/example/inference/coinmc.pl}{\texttt{coinmc.pl}})
\begin{verbatim}
?- mc_sample(heads(coin),1000,Prob,[]).
\end{verbatim}
that samples \verb|heads(coin)| 1000 times and returns the
estimated probability that a sample is true (i.e., that a sample succeeds).


Moreover, you can sample arguments of queries with
\begin{verbatim}
mc_sample_arg(:Query:atom,+Samples:int,?Arg:var,-Values:list).
\end{verbatim}
The predicate samples \verb|Query| a number of \verb|Samples| times.
\verb|Arg| should be a variable in \verb|Query|.
The predicate returns in \verb|Values| a list of couples \verb|L-N| where
\verb|L| is the list of values of \verb|Arg| for which \verb|Query|
succeeds in a world sampled at random and \verb|N|
is the number of samples returning that list of values.
If \verb|L| is the empty list, it means that for that
sample the query failed.
If \verb|L| is a list with a
single element, it means that for that sample the query is
determinate.
If, in all couples \verb|L-N|, \verb|L|
is a list with a
single element, it means that the clauses in the program
are mutually exclusive, i.e., that in every sample, only
one clause for each subgoal has the body true. This is one
of the assumptions taken for programs of the PRISM system \cite{DBLP:journals/jair/SatoK01}.
For example
\href{http://cplint.ml.unife.it/example/inference/pcfglr.pl}{\texttt{pcfglr.pl}} and \href{http://cplint.ml.unife.it/example/inference/plcg.pl}{\texttt{plcg.pl}} satisfy this constraint while
 \href{http://cplint.ml.unife.it/example/inference/markov_chain.pl}{\texttt{markov\_chain.pl}} and \href{http://cplint.ml.unife.it/example/inference/var_obj.pl}{\texttt{var\_obj.pl}} don't.
 
\verb|Options| is a list of options, the following are recognised by \verb|mc_sample_arg/5|:
\begin{itemize}
\item \verb|successes(-Successes:int)|
Number of succeses
\item \verb|failures(-Failures:int)|
Number of failueres
\item \verb|bar(-BarChar:dict)|
BarChart is a dict for rendering with c3 as a bar chart with
a bar for the number of successes and a bar for the number
of failures.
\end{itemize}


An example of use of the above predicate is
\begin{verbatim}
?- mc_sample_arg(reach(s0,0,S),50,S,Values,[]).
\end{verbatim}
of \href{http://cplint.ml.unife.it/example/inference/markov_chain.pl}{\texttt{markov\_chain.pl}}
that takes 50 samples of \verb|L| in \verb|findall(S,(reach(s0,0,S),L)|.

You can sample arguments of queries also with
\begin{verbatim}
mc_sample_arg_raw(:Query:atom,+Samples:int,?Arg:var,-Values:list).
\end{verbatim}
that samples \verb|Query| a number of \verb|Samples| times
The predicate returns in \verb|Values| a list of values
of \verb|Arg|  returned as the first answer by \verb|Query|  in
a world sampled at random.
The value is \verb|failure| if the query fails.

The predicate
\begin{verbatim}
mc_sample_arg_first(:Query:atom,+Samples:int,?Arg:var,-Values:list,+Options:list).
\end{verbatim}
samples \verb|Query| a number of \verb|Samples| times
and returns in \verb|Values| a list of couples \verb|V-N| where
\verb|V| is the value of \verb|Arg| returned as the first answer by \verb|Query| in
a world sampled at random and \verb|N| is the number of samples
returning that value.
\verb|V| is failure if the query fails.
\verb|mc_sample_arg_first/5| differs from \verb|mc_sample_arg/5| because the first just computes the first
answer of the query for each sampled world.

\verb|Options| is a list of options, the following are recognised by \verb|mc_sample_arg_first/5|:
\begin{itemize}
\item \verb|bar(-BarChar:dict)|
BarChart is a dict for rendering with c3 as a bar chart with
a bar for the number of successes and a bar for the number
of failures.
\end{itemize}


The predicate
\begin{verbatim}
mc_sample_arg_one(:Query:atom,+Samples:int,?Arg:var,-Values:list,+Options:list)
\end{verbatim}
 samples \verb|Query| a number of \verb|Samples| times
and returns in \verb|Values| a list of couples \verb|V-N| where
\verb|V| is a value sampled with uniform probability from those returned
by \verb|Query| in a world sampled at random and \verb|N| is the number of samples
returning that value.
\verb|V| is failure if the query fails.

\verb|Options| is a list of options, the following are recognised by \verb|mc_sample_arg_one/5|:
\begin{itemize}
\item \verb|bar(-BarChar:dict)|
BarChart is a dict for rendering with c3 as a bar chart with
a bar for the number of successes and a bar for the number
of failures.
\end{itemize}


Finally, you can compute expectations with
\begin{verbatim}
mc_expectation(:Query:atom,+N:int,?Arg:var,-Exp:float).
\end{verbatim}
that computes the expected value of \verb|Arg| in \verb|Query| by
sampling.
It takes \verb|N| samples of \verb|Query| and sums up the value of \verb|Arg| for
each sample. The overall sum is divided by \verb|N| to give \verb|Exp|.

An example of use of the above predicate is
\begin{verbatim}
?- mc_expectation(eventually(elect,T),1000,T,E).
\end{verbatim}
of \href{http://cplint.ml.unife.it/example/inference/pctl_slep.pl}{\texttt{pctl\_slep.pl}}
that returns in \verb|E| the expected value of \verb|T| by taking 1000 samples.

\subsubsection{Drawing BDDs}

With \verb|pita|, you can obtain the BDD for a query with the predicates
\begin{verbatim}
bdd_dot_file(:Query:atom,+FileName:string,-Var:list)
bdd_dot_string(:Query:atom,-DotString:string,-Var:list)
\end{verbatim}
The first write the BDD to a file, the latter returns it as a string.
The BDD is represented in the dot format of graphviz.
Solid edges indicate 1-children, dashed edges indicate 0-children and dotted
edges indicate 0-children with negation applied to the sub BDD.
Each level of the BDD is associated to a variable of the form XI\_J indicated on the left:
I indicates the multivalued variable index and J the index of the Boolean variable of rule I.
The hexadecimal number in each node is part of its address in memory and is not significant.
The table \verb|Var| contains the associations between the rule groundings and the
multivalued variables: the first column contains contains the multivalued variable index,
the second column contains the rule index, corresponding
to its position in the program, and the last column contains the list
of constants grounding the rule, each replacing a variable in the order of appearance in the
rule.

The BDD can be drawn in \verb|cplint| on SWISH by using the \verb|graphviz| renderer.


\subsection{Conditional Queries on Discrete Variables}
\label{condq}
The conditional probability of an atom query given another atom evidence can be asked using \verb|pita| with the predicate
\begin{verbatim}
prob(:Query:atom,:Evidence:atom,-Probability:float).
\end{verbatim}
as in
\begin{verbatim}
?- prob(heads(coin),biased(coin),P).
\end{verbatim}
If the query/evidence are non-ground, \verb|prob/3| returns in backtracking ground instantiations together with their probability.

The query and the evidence can be conjunctions of literals (positive or negative).

When using \verb|mcintyre|, you can ask conditional queries with rejection sampling or with Metropolis-Hastings Markov Chain Monte Carlo.
In rejection sampling \cite{von195113}, you first query the evidence and, if the query is successful, query the goal in the same sample, otherwise
the sample is discarded.
In Metropolis-Hastings MCMC, \verb|mcintyre| follows the algorithm proposed in \cite{nampally2014adaptive} (the non adaptive version).
A Markov chain is built by building an initial sample and by generating successor samples.

The initial sample is built by  randomly sampling choices so that the evidence is true. This is done with
a backtracking meta-interpreter that starts with the goal and
randomizes the order in which clauses are selected during the search so that the initial sample is unbiased. Each time the meta-interpreter encounters
a probabilistic choice, it first checks whether a
value has already been sampled, if not, it takes
a sample and records it. If a failure is obtained,
the meta-interpreter backtracks to other clauses but
without deleting samples. Then the goal is queries using
regular MCINTYRE.

A successor sample is obtained by deleting a
fixed number (parameter \verb|Lag|) of sampled probabilistic choices. Then the
evidence is queried using regular MCINTYRE starting with the undeleted choices.
If the query succeeds, the goal is queried using regular MCINTYRE.
The sample is accepted with a probability of $\min\{1,\frac{N_0}{N_1}\}$ where $N_0$ is the number of choices sampled
in the previous sample and $N_1$ is the number of choices sampled in the current sample.
In \cite{nampally2014adaptive} the lag is always 1 but the proof in \cite{nampally2014adaptive} that the above acceptance
probability yields a valid
Metropolis-Hastings algorithm holds also when forgetting more than one
sampled choice, so the lag is
user defined in \verb|cplint|.

Then the number of successes of the query is increased by 1 if the query succeeded in the last accepted
sample. The final probability is given by the number of successes over the total
number of samples.

You can take a given number of sample with rejection sampling using
\begin{verbatim}
mc_rejection_sample(:Query:atom,:Evidence:atom,+Samples:int,-Probability:float,+Options:list).
\end{verbatim}
as in (\href{http://cplint.ml.unife.it/example/inference/coinmc.pl}{\texttt{coinmc.pl}})
\begin{verbatim}
?- mc_rejection_sample(heads(coin),biased(coin),1000,P,[successes(S),failures(F)]).
\end{verbatim}
that takes 1000 samples where \verb|biased(coin)| is true and returns in \verb|S| the number of samples where
\verb|heads(coin)| is true, in \verb|F| the number of samples where \verb|heads(coin)| is false and in \verb|P| the
estimated probability (\verb|S/1000|).

The query and the evidence can be conjunctions of literals.

You can take a given number of sample with Metropolis-Hastings MCMC using
\begin{verbatim}
mc_mh_sample(:Query:atom,:Evidence:atom,+Samples:int,
  -Probability:float,+Options:list).
\end{verbatim}
where \verb|Lag| (that is set with the options, default value 1) is the number of sampled choices to forget before taking a new sample.

\verb|Options| is a list of options, the following are recognised by \verb|mc_mh_sample/5|:
\begin{itemize}
\item \verb|mix(+Mix:int)|
The first Mix samples are discarded (mixing time), default value 0
\item \verb|lag(+Lag:int)|
lag between each sample, Lag sampled choices are forgotten, default value 1
\item \verb|successes(-Successes:int)|
Number of succeses
\item \verb|failures(-Failures:int)|
Number of failueres
\item \verb|bar(-BarChar:dict)|
BarChart is a dict for rendering with c3 as a bar chart with
a bar for the number of successes and a bar for the number
of failures.
\end{itemize}
With \verb|Mix| specified it takes \verb|Mix+Samples| samples and discards the first \verb|Mix|.

For example (\href{http://cplint.ml.unife.it/example/inference/arithm.pl}{\texttt{arithm.pl}})
\begin{verbatim}
?- mc_mh_sample(eval(2,4),eval(1,3),10000,P,[successes(T), failures(F)]).
\end{verbatim}
takes 10000 accepted samples and returns in \verb|T| the number of samples where
\verb|eval(2,4)| is true, in \verb|F| the number of samples where \verb|eval(2,4)| is false and in \verb|P| the
estimated probability (\verb|T/10000|).


Moreover, you can sample arguments of queries with rejection sampling and Metropolis-Hastings MCMC using
\begin{verbatim}
mc_rejection_sample_arg(:Query:atom,:Evidence:atom,
  +Samples:int,?Arg:var,-Values:list,+Options:list).
\end{verbatim}
\verb|Options| is a list of options, the following are recognised by \verb|mc_rejection_sample_arg/6|:
\begin{itemize}
\item \verb|successes(-Successes:int)|
Number of succeses
\item \verb|failures(-Failures:int)|
Number of failueres
\item \verb|bar(-BarChar:dict)|
BarChart is a dict for rendering with c3 as a bar chart with
a bar for the number of successes and a bar for the number
of failures.
\end{itemize}
\begin{verbatim}
mc_mh_sample_arg(:Query:atom,:Evidence:atom,
  +Samples:int,?Arg:var,-Values:list,+Options:list).
\end{verbatim}
\verb|Options| is a list of options, the following are recognised by \verb|mc_mh_sample_arg/6|:
\begin{itemize}
\item \verb|mix(+Mix:int)|
The first Mix samples are discarded (mixing time), default value 0
\item \verb|lag(+Lag:int)|
lag between each sample, Lag sampled choices are forgotten, default value 1
\item \verb|successes(-Successes:int)|
Number of succeses
\item \verb|failures(-Failures:int)|
Number of failueres
\item \verb|bar(-BarChar:dict)|
BarChart is a dict for rendering with c3 as a bar chart with
a bar for the number of successes and a bar for the number
of failures.
\end{itemize}
that return the distribution of values for \verb|Arg| in \verb|Query| in \verb|Samples| of
\verb|Query| given that \verb|Evidence| is true. \verb|Mix| indicates the
number of mixing samples.
The predicate returns in \verb|Values| a list of couples \verb|L-N| where
\verb|L| is the list of values of \verb|Arg| for which \verb|Query|
succeeds in a world sampled at random where \verb|Evidence| is true and \verb|N|
is the number of samples returning that list of values.

An example of use of the above predicates is
\begin{verbatim}
?- mc_mh_sample_arg(eval(2,Y),eval(1,3),1000,Y,V,[]).
\end{verbatim}
of \href{http://cplint.ml.unife.it/example/inference/arithm.pl}{\texttt{arithm.pl}}.

Finally, you can compute expectations with
\begin{verbatim}
mc_expectation(:Query:atom,+N:int,?Arg:var,-Exp:float).
\end{verbatim}
that computes the expected value of \verb|Arg| in \verb|Query| by
sampling.
It takes \verb|N| samples of \verb|Query| and sums up the value of \verb|Arg| for
each sample. The overall sum is divided by \verb|N| to give \verb|Exp|.

An example of use of the above predicate is
\begin{verbatim}
?- mc_expectation(eventually(elect,T),1000,T,E).
\end{verbatim}
of \href{http://cplint.ml.unife.it/example/inference/pctl_slep.pl}{\texttt{pctl\_slep.pl}}
that returns in \verb|E| the expected value of \verb|T| by taking 1000 samples.

To compute conditional expectations, use
\begin{verbatim}
mc_mh_expectation(:Query:atom,:Evidence:atom,+N:int,
  +Lag:int,?Arg:var,-Exp:float).
\end{verbatim}
\verb|Options| is a list of options, the following are recognised by \verb|mc_mh_expectation/6|:
\begin{itemize}
\item \verb|mix(+Mix:int)|
The first Mix samples are discarded (mixing time), default value 0
\item \verb|lag(+Lag:int)|
lag between each sample, Lag sampled choices are forgotten, default value 1
\end{itemize}
\begin{verbatim}
mc_rejection_expectation(:Query:atom,:Evidence:atom,+N:int,
  ?Arg:var,-Exp:float).
\end{verbatim}
as in
\begin{verbatim}
?- mc_mh_expectation(eval(2,Y),eval(1,3),1000,1,Y,E).
\end{verbatim}
of \href{http://cplint.ml.unife.it/example/inference/arithm.pl}{\texttt{arithm.pl}}
that computes the expectation of argument \verb|Y| of \verb|eval(2,Y)| given that
\verb|eval(1,3)| is true by taking 1000 samples using Metropolis-Hastings MCMC.

\subsection{Conditional Queries on Continuous Variables}
\label{condqcont}

When you have continuous random variables, you may be interested in
sampling arguments of goals representing continuous random variables.
In this way you can build a probability density of the sampled argument.
When you do not have evidence or you have evidence on atoms not depending
on continuous random variables, you can use the above predicates for sampling
arguments.

For example
\begin{verbatim}
?- mc_sample_arg(value(0,X),1000,X,L,[]).
\end{verbatim}
from (\href{http://cplint.ml.unife.it/example/inference/gauss_mean_est.pl}{\texttt{gauss\_mean\_est.pl}})) samples 1000 values for \verb|X| in
\verb|value(0,X)| and returns them in \verb|L|.

When you have evidence on ground atoms that have continuous values as
arguments, you cannot use rejection sampling or Metropolis-Hastings,
as the probability of the evidence is 0. For example,
the probability of sampling a specific value from a Gaussian is 0.
Continuous variables have probability densities instead of distributions as
discrete variables.
In this case, you can use likelihood weighting or particle filtering \cite{fung1990weighing,koller2009probabilistic,Nitti2016} to obtain samples of
continuous arguments of a goal.

For each sample to be taken, likelihood weighting
uses a meta-interpreter to find a sample where
the goal is true, randomizing the choice of
clauses when more than one resolves with the goal
in order to obtain an unbiased sample.
This meta-interpreter is similar to the one used
to generate the first sample in Metropolis-Hastings.

Then a different meta-interpreter is used to evaluate
the weight of the sample.
This meta-interpreter starts with the evidence as the query and a weight of 1. Each time the meta-interpreter encounters
a probabilistic choice over a continuous variable, it first checks whether a
value has already been sampled.
If so, it computes the probability density of the sampled value and multiplies the weight by it. If the value has not been sampled, it takes a sample and records it,
leaving the weight unchanged.
In this way, each sample in the result has a weight that is 1 for the prior distribution and that may be different from the posterior distribution,
reflecting the influence of evidence.

In particle filtering, the evidence is a list of atoms. Each sample is weighted by the
likelihood of an element of the evidence and constitutes a particle.
After weighting, particles are resampled and the next element of the evidence
is considered.

 The predicate
\begin{verbatim}
mc_lw_sample(:Query:atom,:Evidence:atom,+Samples:int,-Prob:float)
\end{verbatim}
samples \verb|Query|  a number of \verb|Samples| times given that \verb|Evidence|
(a conjunction of atoms is allowed here). is true.
The predicate returns in \verb|Prob| the probability that the query is true.
It performs likelihood weighting: each sample is weighted by the
likelihood of evidence in the sample.
For example
\begin{verbatim}
?- mc_lw_sample(nation(a),student_gpa(4.0),1000,PPost).
\end{verbatim}
from \href{http://cplint.ml.unife.it/example/inference/indian_gpa.pl}{\texttt{indian\_gpa.pl}} samples 1000 the query
\verb|nation(a)| given that \verb|student_gpa(4.0)| has been observed.


 The predicate
\begin{verbatim}
mc_lw_sample_arg(:Query:atom,:Evidence:atom,+N:int,?Arg:var,-ValList)
\end{verbatim}
returns in \verb|ValList| a list of couples \verb|V-W| where \verb|V| is a value of \verb|Arg|
for which \verb|Query| succeeds and \verb|W| is the
weight computed by likelihood weighting
according to \verb|Evidence| (a conjunction of atoms is allowed here).
For example
\begin{verbatim}
?- mc_lw_sample_arg(value(0,X),(value(1,9),value(2,8)),100,X,L).
\end{verbatim}
from \href{http://cplint.ml.unife.it/example/inference/gauss_mean_est.pl}{\texttt{gauss\_mean\_est.pl}} samples 100 values for \verb|X| in
\verb|value(0,X)| given that \verb|value(1,9)| and \verb|value(2,8)| have been observed.

You can compute conditional expectations using likelihood weighting with
\begin{verbatim}
mc_lw_expectation(:Query:atom,Evidence:atom,+N:int,?Arg:var,-Exp:float).
\end{verbatim}
that computes the expected value of \verb|Arg| in \verb|Query| given that
\verb|Evidence| is true.
It takes \verb|N| samples of  \verb|Arg| in \verb|Query|, weighting each according
to the evidence, and returns their weighted average.



The predicate
\begin{verbatim}
mc_particle_sample_arg(:Query:atom,+Evidence:list,
  +Samples:int,?Arg:var,-Values:list)
\end{verbatim}
samples argument \verb|Arg| of \verb|Query| using particle filtering
given that
\verb|Evidence|
is true. \verb|Evidence| is a list of goals and \verb|Query| can be either
a single goal or a list of goals.
When \verb|Query| is a single goal, the predicate returns in \verb|Values| a list of couples \verb|V-W| where
\verb|V| is a value of \verb|Arg| for which \verb|Query| succeeds in
a particle in the last set of particles and \verb|W| is the weight of the particle.
For each element of \verb|Evidence|, the particles are obtained by sampling \verb|Query|
in each current particle and weighting the particle by the likelihood of the evidence element.

When \verb|Query| is a list of goals,  \verb|Arg| is a list of variables, one for
each query of \verb|Query| and \verb|Arg| and \verb|Query| must have the same length of \verb|Evidence|.
\verb|Values| is then list of the same length of \verb|Evidence| and each of its
elements is a list of couples \verb|V-W| where
\verb|V| is a value of the corresponding element of \verb|Arg| for which the corresponding element of
\verb|Query| succeeds in
a particle and \verb|W| is the weight of the particle.
For each element of \verb|Evidence|, the particles are obtained by sampling the corresponding element of \verb|Query|
in each current particle and weighting the particle by the likelihood of the evidence element.


For example
\begin{verbatim}
?-[O1,O2,O3,O4]=[-0.133, -1.183, -3.212, -4.586],
mc_particle_sample_arg([kf_fin(1,T1),kf_fin(2,T2),kf_fin(3,T3),kf_fin(4,T4)],
  [kf_o(1,O1),kf_o(2,O2),kf_o(3,O3),kf_o(4,O4)],100,[T1,T2,T3,T4],[F1,F2,F3,F4]).
\end{verbatim}
from \href{http://cplint.ml.unife.it/example/inference/kalman_filter.pl}{\texttt{kalman\_filter.pl}} performs
particle filtering for a Kalman filter with four observations. For each observation, the value of the state
at the same time point is sampled. The list of samples is returned in \verb|[F1,F2,F3,F4]|, with each element
being the sample for a time point.

The predicate
\begin{verbatim}
mc_particle_sample(:Query:atom,:Evidence:list,
  +Samples:int,-Prob:float)
\end{verbatim}
samples \verb|Query|  a number of \verb|Samples| times given that
\verb|Evidence|
is true using particle filtering. \verb|Evidence| is a list of goals.
The predicate returns in \verb|Prob| the probability that the query is true.

You can compute conditional expectations using particle filtering with
\begin{verbatim}
mc_particle_expectation(:Query:atom,Evidence:atom,+N:int,?Arg:var,-Exp:float).
\end{verbatim}
that computes the expected value of \verb|Arg| in \verb|Query| given that
\verb|Evidence| is true.
It uses \verb|N| particles.

\subsection{Causal Inference}
\label{causal}

\verb|pita| and \verb|mcintyre| support causal reasoning, i.e., computing the effect of actions using the
do-calculus \cite{Pea00-book}.

Actions in this setting are represented as literals of action predicates, that must be declared as such
with the directive
\begin{verbatim}
:- action predicate1/arity1,...,predicaten/arityn.
\end{verbatim}
When performing causal reasoning, action literals must be enclosed in the \verb|do/1| functor and included in the evidence conjunction. More than one action can be included (each with in a separate
\verb|do/1| term) and actions and observations can be freely mixed.
All conditional inference goals can be used except those for particle filtering.

For example
\begin{verbatim}
?- prob(recovery,do(drug),P).
\end{verbatim}
from \href{http://cplint.ml.unife.it/example/inference/simpson.swinb}{\texttt{simpson.swinb}}
computes the probability of recovery of a patient given that the action of administering a drug has
been performed.



ml\subsection{Parameters}
The inference modules have a number of parameters in order to control their behavior. They can be set with the directive
\begin{verbatim}
:- set_pita(<parameter>,<value>).
\end{verbatim}
or
\begin{verbatim}
:- set_mc(<parameter>,<value>).
\end{verbatim}
after initialization (\verb|:-pita.| or \verb|:-mc.|) but outside \verb|:-begin/end_lpad.|
The current value can be read with
\begin{verbatim}
?- setting_pita(<parameter>,Value).
\end{verbatim}
or
\begin{verbatim}
?- setting_mc(<parameter>,Value).
\end{verbatim}
from the top-level.
The available parameters common to both \verb|pita| and \verb|mcintyre| are:
\begin{itemize}
\item 
	 \verb|epsilon_parsing|: if (1 - the sum of the probabilities of all the head atoms) is larger than 
    \verb|epsilon_parsing|,
		then \texttt{pita} adds the null event to the head. Default value \texttt{0.00001}.
\item \verb|single_var|: determines how non ground clauses are treated: if \texttt{true}, a single random variable is assigned to the whole non ground clause, 
if \texttt{false}, a different random variable is assigned to every grounding of the clause. Default value \texttt{false}.
\end{itemize}
Moreover, \verb|pita| has the parameters
\begin{itemize}
\item \verb|depth_bound|: if \texttt{true}, the depth of the derivation of the goal is limited to the value of the \texttt{depth} parameter.  Default value \texttt{false}.
\item  \texttt{depth}: maximum depth of derivations when  \verb|depth_bound| is set to \texttt{true}. Default value \texttt{5}.
\item \verb|prism_memoization|: \verb|false|: original prism semantics, \verb|true|: semantics with memoization
\end{itemize}
If \verb|depth_bound| is set to \verb|true|, derivations are depth-bounded so you can query also programs
containing infinite loops, for example programs where queries have an infinite number of explanations. However the probability that is returned is guaranteed only to be a lower bound,
see for example \href{http://cplint.eu/example/inference/markov_chaindb.pl}{\texttt{markov\_chaindb.pl}}

\verb|mcintyre| has the parameters
\begin{itemize}
\item \verb|min_error|: minimal width of the binomial proportion confidence interval for the probability of the query. When the confidence interval for the probability of the query is below \verb|min_error|, 
the computation stops.
 Default value \verb|0.01|.
\item 
\verb|k|:  the number of samples to take before checking whether the the binomial proportion confidence interval is below \verb|min_error|.
Default value \verb|1000|.
\verb|max_samples|: the maximum number of samples to take. This is used when the probability of the
query is very close to 0 or 1. In fact \verb|mcintyre| also checks for the validity of the
the binomial proportion confidence interval: if less than 5 failures or successes are sampled,
even if the width of the confidence interval is less than \verb|min_error|, the computation continues.
This would lead to non-termination in cases where the probability is 0 or 1. 
\verb|max_samples| ensures termination.
 Default value \verb|10e4|.
\item \verb|prism_memoization|: \verb|false|: original prism semantics, \verb|true|: semantics with memoization
\end{itemize}
The example \href{http://cplint.eu/example/inference/markov_chain.pl}{\texttt{markov\_chain.pl}}
shows that \verb|mcintyre| can perform inference in presence of an infinite number of explanations for 
the goal. Differently from \verb|pita|, no depth bound is necessary, as the probability of selecting
the infinite computation branch is 0. However, also \verb|mcintyre| may not terminate if loops not
involving probabilistic predicates are present.

If you want to set the seed of the random number generator, you can use SWI-Prolog predicates \verb|setrand/1| and \verb|getrand/1|, see
\href{http://www.swi-prolog.org/pldoc/doc_for?object=setrand/1}{SWI-Prolog manual}.



\section{Learning}
\label{learning}
The following learning algorithms are available:
\begin{itemize}
\item EMBLEM (EM over Bdds for probabilistic Logic programs Efficient Mining): an implementation of EM for learning parameters that computes expectations directly on BDDs \cite{BelRig11-IDA-IJ}, \cite{BelRig11-CILC11-NC}, \cite{BelRig11-TR}
\item SLIPCOVER (Structure LearnIng of Probabilistic logic programs by searChing OVER the clause space): an algorithm for learning the structure of programs by searching the clause space and the theory space separately \cite{BelRig13-TPLP-IJ}
\item LEMUR (LEarning with a Monte carlo Upgrade of tRee search): an algorithm 
for learning the structure of programs by searching the clase space using 
Monte-Carlo tree search \cite{DiMBelRig15-ML-IJ}
\end{itemize}

\subsection{Input}
To execute the learning algorithms, prepare a Prolog file divided in five parts
\begin{itemize}
\item preamble
\item  background knowledge, i.e., knowledge valid for all interpretations
\item  LPAD/CPL-program for you which you want to learn the parameters (optional)
\item language bias information
\item  example interpretations 
\end{itemize}
The preamble must come first, the order of the other parts can be changed.

For example, consider the Bongard problems of \cite{RaeLae95-ALT95}. 
%The \texttt{pack/cplint/ prolog/examples/learning} folder in SWI-Prolog home contains some example learning files. 
\href{http://cplint.lamping.unife.it/example/learning/bongard.pl}{\texttt{bongard.pl}} and \href{http://cplint.lamping.unife.it/example/learning/bongardkeys.pl}{\texttt{bongardkeys.pl}} represent a Bongard problem for SLIPCOVER.
\href{http://cplint.lamping.unife.it/example/lemur/bongard.pl}{\texttt{bongard.pl}} and \href{http://cplint.lamping.unife.it/example/lemur/bongardkeys.pl}{\texttt{bongardkeys.pl}} represent a Bongard problem for LEMUR.


\subsubsection{Preamble}
In the preamble, the SLIPCOVER library is loaded with (see \href{http://cplint.lamping.unife.it/example/learning/bongard.pl}{\texttt{bongard.pl}}):
\begin{verbatim}
:- use_module(library(slipcover)).
\end{verbatim}
%Then, if you are using your file in cplint on SWISH, you could add
%\begin{verbatim}
%:- if(current_predicate(use_rendering/1)).
%:- use_rendering(c3).
%:- use_rendering(lpad).
%:- endif.
%\end{verbatim}
%if you want a nice representation of the output (in particular, if you want graphs of the ROC and PR curves).
Now you can initialize SLIPCOVER with
\begin{verbatim}
:- sc.
\end{verbatim}
At this point you can start setting parameters for SLIPCOVER such as for example
\begin{verbatim}
:- set_sc(megaex_bottom,20).
:- set_sc(max_iter,2).
:- set_sc(max_iter_structure,5).
:- set_sc(verbosity,1).
\end{verbatim}
We will see later the list of available parameters.


In the preamble, the LEMUR library is loaded with (see \href{http://cplint.lamping.unife.it/example/lemur/bongard.pl}{\texttt{bongard.pl}}):
\begin{verbatim}
:- use_module(library(lemur)).
\end{verbatim}
%Then, if you are using your file in cplint on SWISH, you could add
%\begin{verbatim}
%:- if(current_predicate(use_rendering/1)).
%:- use_rendering(c3).
%:- use_rendering(lpad).
%:- endif.
%\end{verbatim}
%if you want a nice representation of the output (in particular, if you want graphs of the ROC and PR curves).
Now you can initialize LEMUR with
\begin{verbatim}
:- lemur.
\end{verbatim}
At this point you can start setting parameters for LEMUR such as for example
\begin{verbatim}
:- set_lm(verbosity,1).
\end{verbatim}
A parameter that is particularly important for both SLIPCOVER and LEMUR is \verb|verbosity|: if set
to 1, nothing is printed and learning is  fastest, if set to 3 much information is printed and learning is slowest, 2 is in between.
This ends the preamble.



\subsubsection{Background and Initial LPAD/CPL-program}
%
Now you can specify the background knowledge with a 
fact of the form 
\begin{verbatim}
bg(<list of terms representing clauses>).
\end{verbatim}
where the clauses must currently be deterministic.
Alternatively, you can specify a set of clauses by including them in 
a section between
\verb|:- begin_bg.| and \verb|:- end_bg.| For example
\begin{verbatim}
:- begin_bg.
replaceable(gear).
replaceable(wheel).
replaceable(chain).
not_replaceable(engine).
not_replaceable(control_unit).
component(C):-
  replaceable(C).
component(C):-
  not_replaceable(C).
:- end_bg.
\end{verbatim}
from the \href{http://cplint.lamping.unife.it/example/learning/mach.pl}{\texttt{mach.pl}} example.
If you specify both a \verb|bg/1| fact and a section, the clauses of the two will be combined.


Moreover, you can specify an initial program with a fact of the form 
\begin{verbatim}
in(<list of terms representing clauses>).
\end{verbatim}
The initial program is used in parameter learning for providing 
the structure. The indicated parameters do not matter as they are first randomized.
Remember to enclose each clause in parentheses because \verb|:-| has the highest precedence.

For example, \href{http://cplint.lamping.unife.it/example/learning/bongard.pl}{\texttt{bongard.pl}} has the initial program 
\begin{verbatim}
in([(pos:0.197575 :-
       circle(A),
       in(B,A)),
    (pos:0.000303421 :-
       circle(A),
       triangle(B)), 
    (pos:0.000448807 :-
       triangle(A),
       circle(B))]).
\end{verbatim}
%Both facts should be present. If there are no background/input clauses then write \verb|bg([]).|/\verb|in([]).|
Alternatively, you can specify an input program in a section between \verb|:- begin_in.| and \verb|:- end_in.|, as for example
\begin{verbatim}
:- begin_in.
pos:0.197575 :-
  circle(A),
  in(B,A).
pos:0.000303421 :-
  circle(A),
  triangle(B).
pos:0.000448807 :-
  triangle(A),
  circle(B).
:- end_in.
\end{verbatim}
If you specify both a \verb|in/1| fact and a section, the clauses of the two will be combined.



\subsubsection{Language Bias}
%
The language bias part contains the declarations of the input and output predicates.
Output predicates are declared as
\begin{verbatim}
output(<predicate>/<arity>).
\end{verbatim}
and indicate the predicate whose atoms you want to predict. Derivations for the atoms for this predicates in the input data
are built by the system. These are the predicates for which new clauses are generated.

Input predicates are those whose atoms you are not interested in predicting. You can declare closed world input predicates with
\begin{verbatim}
input_cw(<predicate>/<arity>).
\end{verbatim}
For these predicates, the only true atoms are those in the interpretations and those derivable from them using the background knowledge, the clauses in the input/hypothesized program are not used to derive atoms for these predicates. Moreover,   clauses of the background knowledge that define closed world input predicates and that call an output predicate in the body will not be used for deriving examples.

Open world input predicates are declared with
\begin{verbatim}
input(<predicate>/<arity>).
\end{verbatim}
In this case, if a subgoal for such a predicate is encountered when deriving a subgoal for the output predicates, 
both the facts in the interpretations, those derivable from them and the background knowledge, the background clauses and the clauses of the input program are used.

Then, you have to specify the language bias by means of mode declarations in the style of 
\href{http://www.doc.ic.ac.uk/\string ~shm/progol.html}{Progol}.
\begin{verbatim}
modeh(<recall>,<predicate>(<arg1>,...)).
\end{verbatim}
specifies the atoms that can appear in the head of clauses, while
\begin{verbatim}
modeb(<recall>,<predicate>(<arg1>,...)).
\end{verbatim}
specifies the atoms that can appear in the body of clauses.
\texttt{<recall>} can be an integer or \texttt{*}.
\texttt{<recall>} indicates how many atoms for the predicate specification are
retained in the bottom clause during a saturation step. \texttt{*} stands for all those that are found. Otherwise the indicated number is randomly chosen.

For SLIPCOVER, two specialization modes are available: \verb|bottom| and \verb|mode|.
In the first, a bottom clause is built and the literals to be added during 
refinement are taken from it. In the latter, no bottom clause is built and
the literals to be added during refinement are generated 
directly from the mode declarations. 
LEMUR has only specialization \verb|mode|.

Arguments of the form
\begin{verbatim}
+<type>
\end{verbatim}
specifies that the argument should be an input variable of type \texttt{<type>}, i.e., a variable replacing a \texttt{+<type>} argument in the head or a \texttt{-<type>} argument in a preceding literal in the current hypothesized clause.

Another argument form is
\begin{verbatim}
-<type>
\end{verbatim}
for specifying that the argument should be a output variable of type \texttt{<type>}. 
Any variable can replace this argument, either input or output.
The only constraint on output variables is that those in the head of the current hypothesized 
clause must appear as output variables in an atom of the body.

Other forms are
\begin{verbatim}
#<type>
\end{verbatim}
for specifying an argument which should be replaced by a constant of type \texttt{<type>} in the bottom clause but should not be used for replacing input variables of the following literals when building the bottom clause or 
\begin{verbatim}
-#<type>
\end{verbatim}
for specifying an argument which should be replaced by a constant of type \texttt{<type>} in the bottom clause and that should be used for replacing input variables of the following literals when building the bottom clause. 
%\verb|#| and \verb|-#| differ only in the creation of the bottom clause.
\begin{verbatim}
<constant>
\end{verbatim}
for specifying a constant.

Note that arguments of the form
\verb|#<type>| \verb|-#<type>| are not available in 
specialization mode \verb|mode|, if you want constants to appear in 
the literals you have to indicate them one by one in the mode declarations.


An example of language bias for the Bongard domain is
\begin{verbatim}
output(pos/0).

input_cw(triangle/1).
input_cw(square/1).
input_cw(circle/1).
input_cw(in/2).
input_cw(config/2).

modeh(*,pos).
modeb(*,triangle(-obj)).
modeb(*,square(-obj)).
modeb(*,circle(-obj)).
modeb(*,in(+obj,-obj)).
modeb(*,in(-obj,+obj)).
modeb(*,config(+obj,-#dir)).
\end{verbatim}
SLIPCOVER and LEMUR also require facts for the \verb|determination/2| Aleph-style predicate that indicate which predicates can appear in the body of clauses. 
For example
\begin{verbatim}
determination(pos/0,triangle/1).
determination(pos/0,square/1).
determination(pos/0,circle/1).
determination(pos/0,in/2).
determination(pos/0,config/2).
\end{verbatim}
state that \verb|triangle/1| can appear in the body of clauses for \verb|pos/0|.

SLIPCOVER and LEMUR also allow mode declarations of the form
\begin{verbatim}
modeh(<r>,[<s1>,...,<sn>],[<a1>,...,<an>],[<P1/Ar1>,...,<Pk/Ark>]). 
\end{verbatim}
These mode declarations are used to generate clauses with more than two head atoms. In them, \verb|<s1>,...,<sn>| are schemas,  \verb|<a1>,...,<an>| are atoms such that \verb|<ai>| is obtained from $\verb|<si>|$ by replacing placemarkers with variables, 
\verb|<Pi/Ari>| are the predicates admitted in the body. \verb|<a1>,...,<an>| are used to indicate which variables should be shared by the atoms in the head.
An example of such a mode declaration (from \texttt{uwcselearn.pl}) is
\begin{verbatim}
modeh(*,
[advisedby(+person,+person),tempadvisedby(+person,+person)],
[advisedby(A,B),tempadvisedby(A,B)],
[professor/1,student/1,hasposition/2,inphase/2,
publication/2,taughtby/3,ta/3,courselevel/2,yearsinprogram/2]).
\end{verbatim}
%
If you want to specify negative literals for addition in the body of clauses,
you should define a new predicate in the background as in
\begin{verbatim}
not_worn(C):-
  component(C),
  \+ worn(C).
one_worn:-
  worn(_).
none_worn:-
  \+ one_worn.
\end{verbatim}
from \href{http://cplint.lamping.unife.it/example/learning/mach.pl}{\texttt{mach.pl}} and add the new predicate in a \verb|modeb/2| fact
\begin{verbatim}
modeb(*,not_worn(-comp)).
modeb(*,none_worn).
\end{verbatim}
Note that successful negative literals do not instantiate the variables, so if you want
a variable appearing in a negative literal to be an output variable you must instantiate 
before calling the negative literals.
The new predicates must also be declared as input
\begin{verbatim}
input_cw(not_worn/1).
input_cw(none_worn/0).
\end{verbatim}
Lookahead can also be specified with facts of the form
\begin{verbatim}
lookahead(<literal>,<list of literals>).
\end{verbatim}
In this case when a literal matching \verb|<literal>| is added to the body of clause during refinement, then also
the literals matching \verb|<list of literals>| will be added.
An example of such declaration (from \href{http://cplint.lamping.unife.it/example/learning/muta.pl}{\texttt{muta.pl}}) is
\begin{verbatim}
lookahead(logp(_),[(_=_))]).
\end{verbatim}
Note that
\verb|<list of literals>| is copied with \verb|copy_term/2| before matching, so
variables in common between \verb|<literal>| and \verb|<list of literals>|
may not be in common in the refined clause.

It is also possible to specify that a literal can only be added together with 
other literals with facts of the form 
\begin{verbatim}
lookahead_cons(<literal>,<list of literals>).
\end{verbatim}
In this case \verb|<literal>| is added to the body of clause during refinement only together with
literals matching \verb|<list of literals>|.
An example of such declaration is
\begin{verbatim}
lookahead_cons(logp(_),[(_=_))]).
\end{verbatim}
Also here 
\verb|<list of literals>| is copied with \verb|copy_term/2| before matching, so
variables in common between \verb|<literal>| and \verb|<list of literals>|
may not be in common in the refined clause.

Moreover, we can specify lookahead with
\begin{verbatim}
lookahead_cons_var(<literal>,<list of literals>).
\end{verbatim}
In this case \verb|<literal>| is added to the body of clause during refinement only together with
literals matching \verb|<list of literals>| and \verb|<list of literals>| is not copied before matching, so
variables in common between \verb|<literal>| and \verb|<list of literals>|
are in common also in the refined clause. This is allowed only with
\verb|specialization| set to \verb|bottom|.
An example of such declaration is
\begin{verbatim}
lookahead_cons_var(logp(B),[(B=_))]).
\end{verbatim}

\subsubsection{Example Interpretations}
The last part of the file contains the data.
You can specify data with two modalities:
models and keys.
In the models type, you specify an example model (or interpretation or megaexample) as a list of Prolog facts initiated by 
\verb|begin(model(<name>)).| and terminated by \verb|end(model(<name>)).| as in
\begin{verbatim}
begin(model(2)).
pos.
triangle(o5).
config(o5,up).
square(o4).
in(o4,o5).
circle(o3).
triangle(o2).
config(o2,up).
in(o2,o3).
triangle(o1).
config(o1,up).
end(model(2)).
\end{verbatim}
The interpretations may contain a fact of the form
\begin{verbatim}
prob(0.3).
\end{verbatim}
assigning a probability (0.3 in this case) to the interpretations. If this is omitted, the probability of each interpretation is considered equal to $1/n$ where $n$ is the total number of interpretations. \verb|prob/1| can be used to set a different multiplicity for the interpretations.

The facts in the interpretation are loaded in SWI-Prolog database by adding an extra initial argument equal to the name of the model.
After each interpretation is loaded, a fact of the form \verb|int(<id>)| is asserted, where \verb|id| is the name of the interpretation. This can be used in
order to retrieve the list of interpretations.

Alternatively, with the keys modality, you can directly write the facts and the first argument will be interpreted as a model identifier. The above interpretation in the keys modality is
\begin{verbatim}
pos(2).
triangle(2,o5).
config(2,o5,up).
square(2,o4).
in(2,o4,o5).
circle(2,o3).
triangle(2,o2).
config(2,o2,up).
in(2,o2,o3).
triangle(2,o1).
config(2,o1,up).
\end{verbatim}
which is contained in the \href{http://cplint.lamping.unife.it/example/learning/bongardkeys.pl}{\texttt{bongardkeys.pl}}
This is also how model \verb|2| above is stored in SWI-Prolog database.
The two modalities, models and keys, can be mixed in the same file.
Facts for \verb|int/1| are not asserted for interpretations in the key
modality but can be added by the user explicitly.


Note that you can add background knowledge that is not probabilistic directly to the file writing the clauses taking into account the model argument. For example (\texttt{carc.pl})
contains
\begin{verbatim}
connected(_M,Ring1,Ring2):-
  Ring1 \= Ring2,
  member(A,Ring1),
  member(A,Ring2), !.

symbond(Mod,A,B,T):- bond(Mod,A,B,T).
symbond(Mod,A,B,T):- bond(Mod,B,A,T).
\end{verbatim}
where the first argument of all the atoms is the model.

Example \href{http://cplint.lamping.unife.it/example/learning/registration.pl}{\texttt{registration.pl}} contains for example
\begin{verbatim}
party(M,P):-
  participant(M,_, _, P, _).
\end{verbatim}
that defines intensionally the target predicate \verb|party/1|. Here \verb|M| is the model and \verb|participant/4| is defined in the interpretations.
You can also define intensionally the negative examples with
\begin{verbatim}
neg(party(M,yes)):- party(M,no).
neg(party(M,no)):- party(M,yes).
\end{verbatim}
Then you must indicate how the examples are divided in folds with facts of the form:
\verb|fold(<fold_name>,<list of model identifiers>)|, as for example
\begin{verbatim}
fold(train,[2,3,...]).
fold(test,[490,491,...]).
\end{verbatim}
As the input file is a Prolog program, you can define intensionally the folds as in
\begin{verbatim}
fold(all,F):-
  findall(I,int(I),F).
\end{verbatim}
\verb|fold/2| is dynamic so you can also write (\href{http://cplint.lamping.unife.it/example/learning/registration.pl}{\texttt{registration.pl}})
\begin{verbatim}
:- fold(all,F),
   sample(4,F,FTr,FTe),
   assert(fold(rand_train,FTr)),
   assert(fold(rand_test,FTe)).
\end{verbatim}
which however must be inserted after the input interpretations otherwise the facts for \verb|int/1| will not be available and
the fold \verb|all| would be empty. This command uses  \verb|sample(N,List,Sampled,Rest)| exported from \verb|slipcover| that samples \verb|N| elements from \verb|List| and returns the sampled elements in \verb|Sampled| and the rest in \verb|Rest|. If \verb|List| has \verb|N| elements or less, \verb|Sampled| is equal to \verb|List| 
and \verb|Rest| is empty.

\subsection{Commands}
\subsubsection{Parameter Learning}
To execute EMBLEM, prepare an input file in the editor panel as indicated above 
and call
\begin{verbatim}
?- induce_par(<list of folds>,P).
\end{verbatim}
where \verb|<list of folds>| is a list of the folds for training and
\verb|P| will contain the input program with updated parameters.

For example \href{http://cplint.lamping.unife.it/example/bongard.pl}{\texttt{bongard.pl}}, you can 
perform parameter learning on the \verb|train| fold with 
\begin{verbatim}
?- induce_par([train],P).
\end{verbatim}


\subsubsection{Structure Learning}
To execute SLIPCOVER,
prepare an input file in the editor panel as indicated above 
and call
\begin{verbatim}
?- induce(<list of folds>,P).
\end{verbatim}
where \verb|<list of folds>| is a list of the folds for training and
\verb|P| will contain the learned program.

For example \href{http://cplint.lamping.unife.it/example/learning/bongard.pl}{\texttt{bongard.pl}}, you can perform structure learning on the \verb|train| fold with 
\begin{verbatim}
?- induce([train],P).
\end{verbatim}
A program can also be tested on a test set with \verb|test/7| or \verb|test_prob/6| as
described above.

Between two executions of \verb|induce/2| you should exit SWI-Prolog to have a 
clean database.

To execute LEMUR,
prepare an input file in the editor panel as indicated above 
and call
\begin{verbatim}
?- induce_lm(<list of folds>,P).
\end{verbatim}
where \verb|<list of folds>| is a list of the folds for training and
\verb|P| will contain the learned program.

For example \href{http://cplint.lamping.unife.it/example/lemur/bongard.pl}{\texttt{bongard.pl}}, you can perform structure learning on the \verb|train| fold with 
\begin{verbatim}
?- induce_lm([train],P).
\end{verbatim}
A program can also be tested on a test set with \verb|test_lm/7| or \verb|test_prob_lm/6| that are LEMUR versions of the SLIPCOVER test 
predicates
described above.

Between two executions of \verb|induce_lm/2| you should exit SWI-Prolog to have a 
clean database.

\subsubsection{Testing}


A program can also be tested on a test set with
\begin{verbatim}
?- test(<program>,<list of folds>,LL,AUCROC,ROC,AUCPR,PR).
\end{verbatim}
or
\begin{verbatim}
?- test_prob(<program>,<list of folds>,LL,NPos,NNeg,ExampleList).
\end{verbatim}
where \verb|<program>| is a list of terms representing clauses and
\verb|<list of folds>| is a list of folds.

\verb|test/7| returns the log likelihood of the test examples in \verb|LL|, the Area Under the ROC curve in \verb|AUCROC|, a dictionary containing the list of points (in the form of Prolog pairs \verb|x-y|) of the ROC curve in \verb|ROC|,
the Area Under the PR curve in \verb|AUCPR|, a dictionary containing the list of points of the PR curve in \verb|PR|.

\verb|test_prob/6| returns the log likelihood of the test examples in \verb|LL|, the numbers of positive and negative examples in \verb|NPos| and \verb|NNeg| and the list 
\verb|ExampleList| containing couples \verb|Prob-Ex| where \verb|Ex| is \verb|a| for \verb|a| a positive example and \verb|\+(a)| for \verb|a| a negative example
and \verb|Prob| is the probability of example \verb|a|.


For example, to test on fold \verb|test| the program learned on fold \verb|train| you can run the query
\begin{verbatim}
?- induce_par([train],P),
   test(P,[test],LL,AUCROC,ROC,AUCPR,PR).
\end{verbatim}
Or you can test the input program on the fold \verb|test| with
\begin{verbatim}
?- in(P),
   test(P,[test],LL,AUCROC,ROC,AUCPR,PR).
\end{verbatim}
In \verb|cplint| on SWISH, by including
\begin{verbatim}
:- use_rendering(c3).
:- use_rendering(lpad).
\end{verbatim}
in the code before \verb|:- sc.| the curves will be shown as graphs using C3.js and the output program will be pretty printed.

You can also draw the curves in \texttt{cplint}  on SWISH using R by loading library
\texttt{cplint\_r}  with
\begin{verbatim}
:- use_module(library(cplint_r)).
\end{verbatim}
and using the predicate
\begin{verbatim}
compute_areas_diagrams_r(+ExampleList:list,-AUCROC:float,-AUCPR:float) is det
\end{verbatim}
that takes as input a list \verb|ExampleList| of pairs probability-literal of the form that is returned by
\verb|test_prob/6|.

\subsection{Commands}
\subsubsection{Parameter Learning}
To execute EMBLEM, prepare an input file as indicated above,
 load it into SWI-Prolog and execute
\begin{verbatim}
?- induce_par(<list of folds>,P).
\end{verbatim}
where \verb|<list of folds>| is a list of the folds for training and
\verb|P| will contain the input program with updated parameters.

For example \href{http://cplint.lamping.unife.it/example/learning/bongard.pl}{\texttt{bongard.pl}}, you can load it into SWI-Prolog
with
\begin{verbatim}
?- [bongard].
\end{verbatim}
and perform parameter learning on the \verb|train| fold with 
\begin{verbatim}
?- induce_par([train],P).
\end{verbatim}
A program can also be tested on a test set with
\begin{verbatim}
?- test(<program>,<list of folds>,LL,AUCROC,ROC,AUCPR,PR).
\end{verbatim}
where \verb|<program>| is a list of terms representing clauses and
\verb|<list of folds>| is a list of folds.
This returns the log likelihood of the test examples in \verb|LL|, the Area Under the ROC curve in \verb|AUCROC|, a dictionary containing the list of points (in the form of Prolog pairs \verb|x-y|) of the ROC curve in \verb|ROC|,
the Area Under the PR curve in \verb|AUCPR|, a dictionary containing the list of points of the PR curve in \verb|PR|.

For example, to test on fold \verb|test| the program learned on fold \verb|train| you can run the query
\begin{verbatim}
?- induce_par([train],P),
   test(P,[test],LL,AUCROC,ROC,AUCPR,PR).
\end{verbatim}
Or you can test the input program on the fold \verb|test| with
\begin{verbatim}
?- in(P),
test(P,[test],LL,AUCROC,ROC,AUCPR,PR).
\end{verbatim}

\subsubsection{Structure Learning}
To execute SLIPCOVER,
prepare an input file as indicated above, load it into SWI-Prolog
and call
\begin{verbatim}
?- induce(<list of folds>,P).
\end{verbatim}
where \verb|<list of folds>| is a list of the folds for training and
\verb|P| will contain the learned program.

For example \href{http://cplint.lamping.unife.it/example/learning/bongard.pl}{\texttt{bongard.pl}}, you can load it into SWI-Prolog
with
\begin{verbatim}
?- [bongard].
\end{verbatim}
and perform structure learning on the \verb|train| fold with 
\begin{verbatim}
?- induce([train],P).
\end{verbatim}
A program can also be tested on a test set with \verb|test/7| as
described above.

\subsection{Parameters}
Parameters are set with  commands of the form
\begin{verbatim}
:- set_sc(<parameter>,<value>).
\end{verbatim}
The available parameters are:
\begin{itemize}
\item \verb|specialization|: (values: \verb|{bottom,mode}|,  default value: \texttt{bottom}) specialization mode. 
\item \verb|depth_bound|: (values: \verb|{true,false}|,  default value: \texttt{true}) if \texttt{true}, the depth of the derivation of the goal is limited to the value of the \texttt{depth} parameter. 
\item \verb|depth| (values: integer, default value: 2): depth of derivations if  \verb|depth_bound|  is set to \verb|true|
\item \verb|single_var| (values: \verb|{true,false}|, default value: \verb|false|): if set to \verb|true|, there is a random variable for each clause, instead of a different random variable for each grounding of each clause
\item \verb|epsilon_em| (values: real, default value: 0.1): if the difference in the log likelihood in two successive parameter EM iteration is smaller
than \verb|epsilon_em|, then EM stops 
\item \verb|epsilon_em_fraction| (values: real, default value: 0.01): if the difference in the log likelihood in two successive parameter EM iteration is smaller
than \verb|epsilon_em_fraction|*(-current log likelihood), then EM stops
\item \verb|iter| (values: integer, defualt value: 1): maximum number of iteration of EM parameter learning. If set to -1, no maximum number of iterations is imposed
\item \verb|iterREF| (values: integer, defualt value: 1, valid for  
 SLIPCOVER):
 maximum number of iteration of EM parameter learning for refinements. If set to -1, no maximum number of iterations is imposed.
\item \verb|random_restarts_number| (values: integer, default value: 1, valid for EMBLEM and SLIPCOVER): number of random restarts of parameter EM learning
\item \verb|random_restarts_REFnumber| (values: integer, default value: 1, valid for  SLIPCOVER): number of random restarts of parameter EM learning for refinements
\item \verb|setrand| (values: rand(integer,integer,integer)): seed for the random functions, see SWI-Prolog manual for the allowed values
\item \verb|logzero| (values: negative real, default value $\log(0.000001)$): value assigned to $\log 0$
\item \verb|max_iter| (values: integer, default value: 10, valid for  SLIPCOVER): number of interations of beam search
\item \verb|max_var| (values: integer, default value: 4, valid for 
SLIPCOVER): maximum number of distinct variables in a clause
\item \verb|beamsize|  (values: integer, default value: 100, valid for SLIPCOVER): size of the beam 
\item \verb|megaex_bottom| (values: integer, default value: 1, valid for SLIPCOVER): number of mega-examples on which to build the bottom clauses
\item \verb|initial_clauses_per_megaex| (values: integer, default value: 1, valid for SLIPCOVER): 
 number of bottom clauses to build for each mega-example (or 
 model or interpretation)
\item \verb|d| (values: integer, default value: 1, valid for SLIPCOVER): 
 number of saturation steps when building the bottom clause
\item \verb|max_iter_structure| (values: integer, default value: 10000, valid for SLIPCOVER): 
maximum  number of theory search iterations
\item \verb|background_clauses| (values: integer, default value: 50, valid for SLIPCOVER): 
 maximum numbers of background clauses
\item \verb|maxdepth_var| (values: integer, default value: 2, valid for SLIPCOVER): maximum depth of
variables in clauses (as defined in \cite{DBLP:journals/ai/Cohen95}).
\item \verb|neg_ex| (values:  \verb|given|, \verb|cw|, default value: \verb|cw|): if  set to \verb|given|, the negative examples in testing
are taken from the test folds interpretations, i.e., those examples \verb|ex| stored as \verb|neg(ex)|; if set to \verb|cw|, the negative examples are generated according to the closed world assumption, i.e., all atoms for target predicates that are not positive examples. The set of all atoms is obtained by collecting the set of constants for each type of the arguments of the target predicate.
\item \verb|verbosity| (values: integer in [1,3], default value: 1): level of verbosity of the algorithms
\end{itemize}


\subsection{Files}
The \texttt{pack/cplint/prolog/examples} folder in SWI-Prolog home contains some example programs. The subfolder \texttt{learning} contains some learning examples.
The \texttt{pack/cplint/doc} folder in SWI-Prolog home contains this manual in latex, html and pdf.


\section{License}
\label{license}



\texttt{cplint} follows the Artistic License 2.0 that you can find in \texttt{cplint} root folder. The copyright is by Fabrizio Riguzzi.
\vspace{3mm}


The library \href{http://vlsi.colorado.edu/\string ~fabio/}{CUDD} for manipulating BDDs has the following license:

\vspace{3mm}

Copyright (c) 1995-2004, Regents of the University of Colorado

All rights reserved.

Redistribution and use in source and binary forms, with or without
modification, are permitted provided that the following conditions
are met:

\begin{itemize}
\item
Redistributions of source code must retain the above copyright
notice, this list of conditions and the following disclaimer.
\item
Redistributions in binary form must reproduce the above copyright
notice, this list of conditions and the following disclaimer in the
documentation and/or other materials provided with the distribution.
\item
Neither the name of the University of Colorado nor the names of its
contributors may be used to endorse or promote products derived from
this software without specific prior written permission.
\end{itemize}
THIS SOFTWARE IS PROVIDED BY THE COPYRIGHT HOLDERS AND CONTRIBUTORS
"AS IS" AND ANY EXPRESS OR IMPLIED WARRAN\-TIES, INCLUDING, BUT NOT
LIMITED TO, THE IMPLIED WARRANTIES OF MERCHANTABILITY AND FITNESS
FOR A PARTICULAR PURPOSE ARE DISCLAIMED. IN NO EVENT SHALL THE
COPYRIGHT OWNER OR CONTRIBUTORS BE LIABLE FOR ANY DIRECT, INDIRECT,
INCIDENTAL, SPECIAL, EXEMPLARY, OR CONSEQUENTIAL DAMAGES (INCLUDING,
BUT NOT LIMITED TO, PROCUREMENT OF SUBSTITUTE GOODS OR SERVICES;
LOSS OF USE, DATA, OR PROFITS; OR BUSINESS INTERRUPTION) HOWEVER
CAU-SED
\\ AND ON ANY THEORY OF LIABILITY, WHETHER IN CONTRACT, STRICT
LIABILITY, OR TORT (INCLUDING NEGLIGENCE OR OTHERWISE) ARISING IN
ANY WAY OUT OF THE USE OF THIS SOFTWARE, EVEN IF ADVISED OF THE
POSSIBILITY OF SUCH DAMAGE.


\bibliographystyle{plain}
\bibliography{bib}

\end{document}
