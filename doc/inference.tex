\section{Inference}
\label{inf}
\texttt{cplint} answers queries using the module \verb|pita|. It performs the program transformation technique of \cite{RigSwi10-ICLP10-IC}. Differently from that work, techniques alternative to tabling and answer subsumption are used.

For answering queries, you have to prepare a Prolog file where you first load \texttt{pita} and then enclose the probabilistic 
clauses in \texttt{:-cplint.} and \texttt{:-end\_cplint.} For example, the coin program above can be stored in \href{http://cplint.lamping.unife.it/example/coin.pl}{\texttt{coin.pl}} as follows
\begin{verbatim}
:- use_module(library(pita)).
:- cplint.

heads(Coin):1/2 ; tails(Coin):1/2:- 
toss(Coin),\+biased(Coin).

heads(Coin):0.6 ; tails(Coin):0.4:- 
toss(Coin),biased(Coin).

fair(Coin):0.9 ; biased(Coin):0.1.

toss(coin).

:- end_cplint.
\end{verbatim}
