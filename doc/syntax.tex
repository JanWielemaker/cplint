\section{Syntax}
\label{syn}
\texttt{cplint} permits the definition of discrete probability distributions and continuous probaility
densities.
\subsection{Discrete Probability Distributions}
\label{discrete}
LPAD and CP-logic programs consist of a set of annotated disjunctive clauses.
Disjunction in the head is represented with a semicolon and atoms in the head are separated from probabilities by a colon. For the rest, the usual syntax of Prolog is used.
A general CP-logic clause has the form
\begin{verbatim}
h1:p1 ; ... ; hn:pn :- b1,...,bm,\+ c1,....,\+ cl
\end{verbatim}
 No parentheses are necessary. The \texttt{pi} are numeric expressions. It is up to the user to ensure that the numeric expressions are legal, i.e. that they sum up to less than one.

If the clause has an empty body, it can be represented like this
\begin{verbatim}
h1:p1 ; ... ; hn:pn.
\end{verbatim}
If the clause has a single head with probability 1, the annotation can be omitted and the clause takes the form of a normal prolog clause, i.e. 
\begin{verbatim}
h1 :- b1,...,bm,\+ c1,...,\+ cl.
\end{verbatim}
stands for 
\begin{verbatim}
h1:1 :- b1,...,bm,\+ c1,...,\+ cl.
\end{verbatim}
The coin example of  \cite{VenVer04-ICLP04-IC} is represented as (file \href{http://cplint.lamping.unife.it/example/inference/coin.cpl}{\texttt{coin.cpl}})
\begin{verbatim}
heads(Coin):1/2 ; tails(Coin):1/2 :- 
  toss(Coin),\+biased(Coin).

heads(Coin):0.6 ; tails(Coin):0.4 :- 
  toss(Coin),biased(Coin).

fair(Coin):0.9 ; biased(Coin):0.1.

toss(coin).
\end{verbatim}
The first clause states that if we toss a coin that is not biased it has equal probability of landing heads and tails. The second states that if the coin is biased it has a slightly higher probability of landing heads. The third states that the coin is fair with probability 0.9 and biased with probability 0.1 and the last clause states that we toss a coin with certainty.

Moreover, the bodies of rules may contain the built-in predicates:
\begin{verbatim}
is/2, >/2, </2, >=/2 ,=</2,
=:=/2, =\=/2, true/0, false/0,
=/2, ==/2, \=/2 ,\==/2, !/0, length/2
\end{verbatim}
The bodies may also contain the following
 library predicates:
\begin{verbatim}
member/2, max_list/2, min_list/2
nth0/3, nth/3, name/2, float/1,
integer/1, var/1, @>/2, 
memberchk/2, select/3, dif/2,
between/3
\end{verbatim}
plus the predicate
\begin{verbatim}
average/2
\end{verbatim}
that, given a list of numbers, computes its arithmetic mean.

The body of rules may also contain the predicate \verb|prob/2| that computes the
probability of an atom, thus allowing nested probability computations.
For example (\href{http://cplint.lamping.unife.it/example/inference/meta.pl}{\texttt{meta.pl}})
\begin{verbatim}
a:0.2:-
  prob(b,P),
  P>0.2.
\end{verbatim}
is a valid rule.

Moreover, the probabilistic annotations can be variables, as in 
(\href{http://cplint.lamping.unife.it/example/inference/flexprob.pl}{\texttt{flexprob.pl}}))
\begin{verbatim}
red(Prob):Prob.

draw_red(R, G):-
  Prob is R/(R + G),
  red(Prob).
\end{verbatim}
Variables in probabilistic annotations must be ground when resolution reaches the end of the body, 
otherwise an exception is raised.

Facts of the form
\begin{verbatim}
A:discrete(Var,D).
\end{verbatim}
where \verb|A| is an atom containg variable \verb|Var| and \verb|D|
is a list of couples \verb|Value:Prob| assigning probability \verb|Prob|
to \verb|Value|. 

\subsection{Continuous Probability Densities}
\label{cont}

\verb|cplint| handles continuous random variables as well with its
sampling inference module.
To specify a probability density on an argument \verb|Var| of an atom
\verb|A| you can used a fact of the form
\begin{verbatim}
A:Density.
\end{verbatim}
where \verb|Density| is a special atom identifying a probability  density on variable \verb|Var|.
Allows \verb|Density| atoms are
\begin{itemize}
\item \verb|uniform(Var,L,U)|: \verb|Var| is uniformly distributed in $[L,U]$
\item \verb|gaussian(Var,Mean,Variance)|: \verb|Var| follows a Gaussian distribution with mean \verb|Mean| and variance \verb|Variance|
\item \verb|dirichlet(Var,Par)|: \verb|Var| is a list of real
numbers following a Dirichlet distribution with $\alpha$ parameters specified
by the list \verb|Par|
\item \verb|gamma(Var,Shape,Scale)|  \verb|Var| follows a gamma distribution 
with shape parameter \verb|Shape| and scale parameter \verb|Scale|.
\end{itemize}
For example
\begin{verbatim}
g(X): gaussian(X,0, 1).
\end{verbatim}
states that argument \verb|X| of \verb|g(X)| follows a Gaussian 
distribution with mean 0 and variance 1.

For example, (\href{http://cplint.lamping.unife.it/example/inference/gaussian_mixture.pl}{\texttt{gaussian\_mixture.pl}})) defines a mixture of two Gaussians:
\begin{verbatim}
heads:0.6;tails:0.4.
g(X): gaussian(X,0, 1).
h(X): gaussian(X,5, 2).
mix(X) :- heads, g(X).
mix(X) :- tails, h(X).
\end{verbatim}
The argument \verb|X| of
\verb|mix(X)| follows a distribution that is a mixture of two Gaussian,
one with mean 0 and variance 1 with probability 0.6 and one with 
mean 5 and variance 2 with probability 0.4.

The parameters of the distribution atoms can be taken from the probabilistic
atom, the example (\href{http://cplint.lamping.unife.it/example/inference/gauss_mean_est.pl}{\texttt{gauss\_mean\_est.pl}}))
\begin{verbatim}
value(I,X) :-
  mean(M),
  value(I,M,X).
mean(M): gaussian(M,1.0, 5.0).
value(_,M,X): gaussian(X,M, 2.0).
\end{verbatim}
states that for an index \verb|I| the continuous variable \verb|X| is 
sampled from a Gaussian whose mean is sampled from a Guassian with mean 1 and
variance 5 and whose variance is 2.

Any operation is allowed on continuous random variables. For example
\href{http://cplint.lamping.unife.it/example/inference/kalman_filter.pl}{\texttt{kalman\_filter.pl}}
\begin{verbatim}
kf(N,O, T) :-
  init(S),
  kf_part(0, N, S,O,T).

kf_part(I, N, S,[V|RO], T) :-
  I < N,
  NextI is I+1,
  trans(S,I,NextS),
  emit(NextS,I,V),
  kf_part(NextI, N, NextS,RO, T).

kf_part(N, N, S, [],S).

trans(S,I,NextS) :-
  {NextS =:= E + S},
  trans_err(I,E).

emit(NextS,I,V) :-
  {NextS =:= V+X},
  obs_err(I,X).

init(S):gaussian(S,0,1).

trans_err(_,E):gaussian(E,0,2).

obs_err(_,E):gaussian(E,0,1).
\end{verbatim}
encodes a Kalman filter. Continuous random variables are involved
in arithmetic expressions (in \verb|trans/3| and \verb|emit/3|). It
is often convenient, as in this case, to use CLP(R) constraints (by
including the directive \verb|:- use_module(library(clpr)).|) as 
in this way the expressions can be used in multiple directions and 
the same program can be used to evaluate both to sample and the weight the sample on the basis
of evidence,
otherwise two different programs have to be written.
In case random variables are not sufficiently instantiated to 
exploit expressions for inferring the values of other variables, 
inference will return an error.

