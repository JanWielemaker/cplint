\subsection{Unconditional Queries}
\label{uncondq}
You can ask the unconditional probability of an atom using \verb|pita| with the predicate
\begin{verbatim}
prob(:Query:atom,-Probability:float)|
\end{verbatim}
as in
\begin{verbatim}
?- prob(heads(coin),P).
\end{verbatim}
If the query is non-ground, \verb|prob/2| returns in backtracking the succesful instantiations together with their probability.

When using \verb|mcintyre|, the predicate for querying is 
\begin{verbatim}
mc_prob(:Query:atom,-Probability:float)
\end{verbatim} as in
\begin{verbatim}
?- mc_prob(heads(coin),P).
\end{verbatim}
With \verb|mcintyre|, you can also take a given number of sample with
\begin{verbatim}
mc_sample(:Query:atom,+Samples:int,-Successes:int,-Failures:int,
  -Probability:float).
\end{verbatim}
as in (\href{http://cplint.lamping.unife.it/example/inference/coinmc.pl}{\texttt{coinmc.pl}})
\begin{verbatim}
?- mc_sample(heads(coin),1000,S,F,P).
\end{verbatim}
that samples \verb|heads(coin)| 1000 times and returns in \verb|S| the number of successes, in \verb|F| the number of failures and in \verb|P| the
estimated probability (\verb|S/1000|).

If you are just interested in the probability, you can use
\begin{verbatim}
mc_sample(:Query:atom,+Samples:int,-Probability:float) 
\end{verbatim}
as in (\href{http://cplint.lamping.unife.it/example/inference/coinmc.pl}{\texttt{coinmc.pl}})
\begin{verbatim}
?- mc_sample(heads(coin),1000,Prob).
\end{verbatim}
that samples \verb|heads(coin)| 1000 times and returns the
estimated probability that a sample is true (i.e., that a sample succeeds).


Moreover, you can sample arguments of queries with
\begin{verbatim}
mc_sample_arg(:Query:atom,+Samples:int,?Arg:var,-Values:list).
\end{verbatim}
The predicate samples \verb|Query| a number of \verb|Samples| times. 
\verb|Arg| should be a variable in \verb|Query|.
The predicate returns in \verb|Values| a list of couples \verb|L-N| where
\verb|L| is the list of values of \verb|Arg| for which \verb|Query|
succeeds in a world sampled at random and \verb|N|
is the number of samples returning that list of values.
If \verb|L| is the empty list, it means that for that
sample the query failed. 
If \verb|L| is a list with a 
single element, it means that for that sample the query is 
determinate. 
If, in all couples \verb|L-N|, \verb|L| 
is a list with a 
single element, it means that the clauses in the program 
are mutually exclusive, i.e., that in every sample, only
one clause for each subgoal has the body true. This is one
of the assumptions taken for programs of the PRISM system \cite{DBLP:journals/jair/SatoK01}.
For example
\href{http://cplint.lamping.unife.it/example/inference/pfcglr.pl}{\texttt{pfcglr.pl}} and \href{http://cplint.lamping.unife.it/example/inference/plcg.pl}{\texttt{plcg.pl}} satisfy this constraint while
 \href{http://cplint.lamping.unife.it/example/inference/markov_chain.pl}{\texttt{markov\_chain.pl}} and \href{http://cplint.lamping.unife.it/example/inference/var_obj.pl}{\texttt{var\_obj.pl}} don't.


An example of use of the above predicate is
\begin{verbatim}
?- mc_sample_arg(reach(s0,0,S),50,S,Values). 
\end{verbatim}
of \href{http://cplint.lamping.unife.it/example/inference/markov_chain.pl}{\texttt{markov\_chain.pl}}
that takes 50 samples of \verb|L| in \verb|findall(S,(reach(s0,0,S),L)|.

You can sample arguments of queries also with
\begin{verbatim}
mc_sample_arg_first(:Query:atom,+Samples:int,?Arg:var,-Values:list).
\end{verbatim}
samples \verb|Query| a number of \verb|Samples| times 
and returns in \verb|Values| a list of couples \verb|V-N| where 
\verb|V| is the value of \verb|Arg| returned as the first answer by \verb|Query| in 
a world sampled at random and \verb|N| is the number of samples
returning that value.
\verb|V| is failure if the query fails.
\verb|mc_sample_arg_first/4| differs from \verb|mc_sample_arg/4| because the first just computes the first
answer of the query for each sampled world.

%Alternatively, you can use
%\begin{verbatim}
%mc_sample_arg_one(:Query:atom,+Samples:int,?Arg:var,-Values:list)
%\end{verbatim}
%that samples \verb|Query| a number of \verb|Samples| times 
%and returns in \verb|Values| a list of couples \verb|V-N| where 
%\verb|V| is a value sampled with uniform probability from those returned 
%by \verb|Query| in a world sampled at random and \verb|N| is the number of samples
%returning that value.
%\verb|V| is failure if the query fails.

Finally, you can compute expectations with 
\begin{verbatim}
mc_expectation(:Query:atom,+N:int,?Arg:var,-Exp:float).
\end{verbatim}
that computes the expected value of \verb|Arg| in \verb|Query| by
sampling.
It takes \verb|N| samples of \verb|Query| and sums up the value of \verb|Arg| for
each sample. The overall sum is divided by \verb|N| to give \verb|Exp|.

An example of use of the above predicate is
\begin{verbatim}
?- mc_expectation(eventually(elect,T),1000,T,E).
\end{verbatim}
of \href{http://cplint.lamping.unife.it/example/inference/pctl_slep.pl}{\texttt{pctl\_slep.pl}}
that returns in \verb|E| the expected value of \verb|T| by taking 1000 samples.

\subsection{Conditional Queries}
\label{condq}
You can ask the conditional probability of an atom given another atom using \verb|pita| with the predicate 
\begin{verbatim}
prob(:Query:atom,:Evidence:atom,-Probability:float)|
\end{verbatim}
as in
\begin{verbatim}
?- prob(heads(coin),biased(coin),P).
\end{verbatim}
If the query/evidence are non-ground, \verb|prob/3| returns in backtracking ground instantiations together with their probability.

When using \verb|mcintyre|, you can ask conditional queries with rejection sampling or with Metropolis-Hastings Markov Chain Monte Carlo.
In rejection sampling, you first query the evidence and, if the query is successful, query the goal in the same sample, otherwise
the sample is discarded.
In Metropolis-Hastings MCMC, \verb|mcintyre| follows the algorithm proposed in \cite{nampally2014adaptive} (the non adaptive version):
after a sample, a number of sampled probabilistic choices are deleted and the others are retained for the next sample.
The sample is accepted with a probability of \verb|min{1,N0/N}| where \verb|N0| is the number of choices sampled
in the previous sample and \verb|N1| is the number of choices sampled in the current sample.


You can take a given number of sample with rejection sampling using
\begin{verbatim}
mc_rejection_sample(:Query:atom,:Evidence:atom,+Samples:int,
  -Successes:int,-Failures:int,-Probability:float).
\end{verbatim}
as in (\href{http://cplint.lamping.unife.it/example/inference/coinmc.pl}{\texttt{coinmc.pl}})
\begin{verbatim}
?- mc_rejection_sample(heads(coin),biased(coin),1000,S,F,P).
\end{verbatim}
that takes 1000 samples where \verb|biased(coin)| is true and returns in \verb|S| the number of samples where 
\verb|heads(coin)| is true, in \verb|F| the number of samples where \verb|heads(coin)| is false and in \verb|P| the
estimated probability (\verb|S/1000|).

You can take a given number of sample with Metropolis-Hastings MCMC using
\begin{verbatim}
mc_mh_sample(:Query:atom,:Evidence:atom,+Samples:int,+Lag:int,
  -Successes:int,-Failures:int,-Probability:float).
\end{verbatim}
where \verb|Lag| is the number of sampled choices to forget before taking a new sample.
For example (\href{http://cplint.lamping.unife.it/example/inference/arithm.pl}{\texttt{arithm.pl}})
\begin{verbatim}
?- mc_mh_sample(eval(2,4),eval(1,3),10000,1,T,F,P).
\end{verbatim}
that take 10000 accepted samples and returns in \verb|T| the number of samples where 
\verb|eval(2,4)| is true, in \verb|F| the number of samples where \verb|eval(2,4)| is false and in \verb|P| the
estimated probability (\verb|T/10000|).


Moreover, you can sample arguments of queries with rejection sampling and Metropolis-Hastings MCMC using
\begin{verbatim}
mc_rejection_sample_arg(:Query:atom,:Evidence:atom,
  +Samples:int,?Arg:var,-Values:list).
mc_mh_sample_arg(:Query:atom,:Evidence:atom,
  +Samples:int,+Lag:int,?Arg:var,-Values:list).
\end{verbatim}
that return the distribution of values for \verb|Arg| in \verb|Query| in \verb|Samples| of
\verb|Query| given that \verb|Evidence| is true.
The predicate returns in \verb|Values| a list of couples \verb|L-N| where
\verb|L| is the list of values of \verb|Arg| for which \verb|Query|
succeeds in a world sampled at random where \verb|Evidence| is true and \verb|N|
is the number of samples returning that list of values.

An example of use of the above predicates is
\begin{verbatim}
?- mc_mh_sample_arg(eval(2,Y),eval(1,3),1000,1,Y,V).
\end{verbatim}
of \href{http://cplint.lamping.unife.it/example/inference/arithm.pl}{\texttt{arithm.pl}}.

Finally, you can compute expectations with 
\begin{verbatim}
mc_expectation(:Query:atom,+N:int,?Arg:var,-Exp:float).
\end{verbatim}
that computes the expected value of \verb|Arg| in \verb|Query| by
sampling.
It takes \verb|N| samples of \verb|Query| and sums up the value of \verb|Arg| for
each sample. The overall sum is divided by \verb|N| to give \verb|Exp|.

An example of use of the above predicate is
\begin{verbatim}
?- mc_expectation(eventually(elect,T),1000,T,E).
\end{verbatim}
of \href{http://cplint.lamping.unife.it/example/inference/pctl_slep.pl}{\texttt{pctl\_slep.pl}}
that returns in \verb|E| the expected value of \verb|T| by taking 1000 samples.

To compute conditional expectations, use
\begin{verbatim}
mc_mh_expectation(:Query:atom,:Evidence:atom,+N:int,
  +Lag:int,?Arg:var,-Exp:float).
\end{verbatim}
as in
\begin{verbatim}
?- mc_mh_expectation(eval(2,Y),eval(1,3),1000,1,Y,E).
\end{verbatim}
of \href{http://cplint.lamping.unife.it/example/inference/arithm.pl}{\texttt{arithm.pl}}
that computes the expectation of argument \verb|Y| of \verb|eval(2,Y)| given that 
\verb|eval(1,3)| is true by taking 1000 samples using Metropolis-Hastings MCMC.