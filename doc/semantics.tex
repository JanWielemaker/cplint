\section{Semantics}
\label{semantics}

The semantics of LPADs for the case of programs without functions symbols can be given
as follows. An LPAD defines a probability distribution over normal logic programs called
\emph{worlds}. A world is obtained from an LPAD by first grounding it, by 
selecting a single head atom for each ground clause and by including in the world
the clause with the selected head atom and the body.
The probability of a world is the product of the probabilities associated to the 
heads selected.
The probability of a ground atom (the query) is given by the sum of the probabilities
of the worlds where the query is true.

If the LPAD contains function symbols, the definition is more complex, see
\cite{DBLP:journals/ai/Poole97,DBLP:journals/jair/SatoK01,Rig15-PLP-IW}.

For the semantics of programs with continuous random variables, see \cite{Nitti2016} that defines distributional clauses.
\verb|cplint| allows more freedom in the use of continuous random variables
in expressions, for example
\href{http://cplint.lamping.unife.it/example/inference/kalman_filter.pl}{\texttt{kalman\_filter.pl}} would not be allowed by distributional clauses.
